% !TEX TS-program = XeLaTeX
% Compilation procedure for this file:
% xelatex Example-IMOS-2.044-iran-unsrt.tex
% bibtex8 -W -c iran-bibtex-cp1256fa Example-IMOS-2.044-iran-unsrt
% xelatex Example-IMOS-2.044-iran-unsrt.tex
% xelatex Example-IMOS-2.044-iran-unsrt.tex

\documentclass[a4paper,10pt]{article}


\usepackage[iran-unsrt]{iran-bibtex}
\usepackage[colorlinks,citecolor=blue,breaklinks=true]{hyperref}

\usepackage[localise,perpagefootnote=on,computeautoilg=on]{xepersian}
\settextfont{Parsi Nevis}





\begin{document}
\title{شیوه‌ی استناددهی ایران
\LTRfootnote{ \tt \textbackslash usepackage[iran-unsrt]\{iran-bibtex\} } }
\author{}
\date{}
\maketitle



\section*{مثال ۲.۴۴: مترجم همراه با نویسنده‌ی کتاب}

زیربخش ۱. سال اثر اصلی، و سال ترجمه هر دو وجود دارند.\\
\cite{بوسکالیا1978-1}\\
\cite{خوشدل1379-1}\\
\cite{tzu6bc-1}\\
\cite{griffith1963-1}\\


زیربخش ۲. تنها سال اثر اصلی وجود دارد.\\
\cite{بوسکالیا1978-2}\\
\cite{خوشدل1379-2}\\
\cite{tzu6bc-2}\\
\cite{griffith1963-2}\\


زیربخش ۳. تنها سال ترجمه وجود دارد.\\
\cite{بوسکالیا1978-3}\\
\cite{خوشدل1379-3}\\
\cite{tzu6bc-3}\\
\cite{griffith1963-3}\\





\bibliography{Example-IMOS-2.044}


\end{document}