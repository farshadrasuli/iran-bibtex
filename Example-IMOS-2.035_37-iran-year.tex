% !TEX TS-program = XeLaTeX
% Compilation procedure for this file:
% xelatex Example-IMOS-2.035_37-iran-year.tex
% bibtex8 -W -c iran-bibtex-cp1256fa Example-IMOS-2.035_37-iran-year
% xelatex Example-IMOS-2.035_37-iran-year.tex
% xelatex Example-IMOS-2.035_37-iran-year.tex

\documentclass[a4paper,10pt]{article}


\usepackage[iran-year]{iran-bibtex}
\usepackage[colorlinks,citecolor=blue,breaklinks=true]{hyperref}

\usepackage[localise,perpagefootnote=on,computeautoilg=on]{xepersian}
\settextfont{Parsi Nevis}





\begin{document}
\title{شیوه‌ی استناددهی ایران
\LTRfootnote{ \tt \textbackslash usepackage[iran-year]\{iran-bibtex\} } }
\author{}
\date{}
\maketitle



\section*{مثالهای ۲.۳۵--۳۷: کتاب با نام مستعار نویسنده}

برای مرجع \cite{یوشیج1383a}، نام مستعار پدیدآورنده وارد شده است؛ در حالی که در مرجع \cite{یوشیج1383b} نام شناسنا‌مه‌ای پدیدآورنده نیز آورده شده است و در مرجع \cite{اسفندیاری1383} از نام شناسنامه‌ای پدیدآورنده استفاده شده و نام مستعار در ادامه نوشته شده است.





\bibliography{Example-IMOS-2.035_37}


\end{document}