% !TEX TS-program = XeLaTeX
% Compilation procedure for this file:
% xelatex Separate-References.tex
% bibtex8 -W -c iran-bibtex-cp1256fa Separate-References
% bibtex8 -W -c iran-bibtex-cp1256fa fa
% bibtex8 -W -c iran-bibtex-cp1256fa en
% xelatex Separate-References.tex
% xelatex Separate-References.tex

\documentclass[a4paper,10pt]{article}


\usepackage[iran]{iran-bibtex}

% use multibib package to separate references
\usepackage{multibib}

\newcites{fa}{مرجع‌های فارسی}
\newcites{en}{مرجع‌های انگلیسی}

\bibliographystylefa{iran}
\bibliographystyleen{iran}


\usepackage[colorlinks,citecolor=blue,breaklinks=true]{hyperref}

\usepackage[localise,perpagefootnote=on,computeautoilg=on]{xepersian}
\settextfont{Parsi Nevis}





\begin{document}
\title{بسته‌ی استناددهی \lr{\bf\sffamily iran-bibtex}}
\author{}
\date{}
\maketitle



\section*{مثال افزوده: جدا کردن مرجعهای فارسی، و انگلیسی}

\citefa{کمالزاده1388}\\
\citefa{مهاجر1398}\\
\citeen{kasper1999}\\
\citeen{suangtho1990}





\bibliographyfa{Separate-References}


\bibliographyen{Separate-References}


\end{document}