% !TEX TS-program = XeLaTeX
% Compilation procedure for this file:
% xelatex Example-IMOS-2.026-iran-unsrt.tex
% bibtex8 -W -c iran-bibtex-cp1256fa Example-IMOS-2.026-iran-unsrt
% xelatex Example-IMOS-2.026-iran-unsrt.tex
% xelatex Example-IMOS-2.026-iran-unsrt.tex

\documentclass[a4paper,10pt]{article}


\usepackage[iran-unsrt]{iran-bibtex}
\usepackage[colorlinks,citecolor=blue,breaklinks=true]{hyperref}

\usepackage[localise,perpagefootnote=on,computeautoilg=on]{xepersian}
\settextfont{Parsi Nevis}





\begin{document}
\title{شیوه‌ی استناددهی ایران
\LTRfootnote{ \tt \textbackslash usepackage[iran-unsrt]\{iran-bibtex\} } }
\author{}
\date{}
\maketitle



\section*{مثال ۲.۲۶: کتاب با یک نویسنده}

مرجع \cite{کمالزاده1388} یک کتاب فارسی با یک نویسنده است. مرجع \cite{kasper1999} یک کتاب غیرفارسی با یک نویسنده است. مرجع \cite{salinger1951} نیز یک کتاب غیرفارسی با یک نویسنده است با این تفاوت که نام فارسی نویسنده‌ی آن وارد شده است.





\bibliography{Example-IMOS-2.026}


\end{document}