% !TEX TS-program = XeLaTeX
% Compilation procedure for this file:
% xelatex Example-IMOS-2.032_33-iran.tex
% bibtex8 -W -c iran-bibtex-cp1256fa Example-IMOS-2.032_33-iran
% xelatex Example-IMOS-2.032_33-iran.tex
% xelatex Example-IMOS-2.032_33-iran.tex

\documentclass[a4paper,10pt]{article}


\usepackage[iran]{iran-bibtex}
\usepackage[colorlinks,citecolor=blue,breaklinks=true]{hyperref}

\usepackage[localise,perpagefootnote=on,computeautoilg=on]{xepersian}
\settextfont{Parsi Nevis}





\begin{document}
\title{شیوه‌ی استناددهی ایران
\LTRfootnote{ \tt \textbackslash usepackage[iran]\{iran-bibtex\} } }
\author{}
\date{}
\maketitle



\section*{مثال ۲.۳۲--۳۳: کتاب با پدیدآونده‌ی ناشناس}

اگر پدیدآورنده‌ی اثری نامشخص باشد، دو راه برای استناد به آن ممکن است: راه نخست آن است که نام پدیدآورنده وارد نشود \cite[مانند][]{هزارویک1378}؛ و راه دیگر آن است که به جای نام پدیدآورنده، «ناشناس» نوشته شود \cite[مانند][]{ناشناس1378,ناشناس1362}. اگر کتاب دارای گردآورنده، یا ویراستار باشد، میتوان آن را زیر نام یکی از این دو نیز فهرست کرد \cite[مانند][]{افراسیابی1378}. با مرجعهای غیرفارسی نیز به همین‌گونه رفتار میشود.
\nocite{*}





\bibliography{Example-IMOS-2.032_33}


\end{document}