% !TEX TS-program = XeLaTeX
% Compilation procedure for this file:
% xelatex Example-IMOS-2.095-iran-plain.tex
% bibtex8 -W -c iran-bibtex-cp1256fa Example-IMOS-2.095-iran-plain
% xelatex Example-IMOS-2.095-iran-plain.tex
% xelatex Example-IMOS-2.095-iran-plain.tex

\documentclass[a4paper,10pt]{article}


\usepackage[iran-plain]{iran-bibtex}
\usepackage[colorlinks,citecolor=blue,breaklinks=true]{hyperref}

\usepackage[localise,perpagefootnote=on,computeautoilg=on]{xepersian}
\settextfont{Parsi Nevis}





\begin{document}
\title{شیوه‌ی استناددهی ایران
\LTRfootnote{ \tt \textbackslash usepackage[iran-plain]\{iran-bibtex\} } }
\author{}
\date{}
\maketitle



\section*{مثال ۲.۹۵: مقدمه‌ای بر یک کتاب}

مرجعهای \cite{معتمدنژاد1376,friedman1994} زیر نام نویسنده‌ی مقدمه مرتب میشوند؛ در حالی که مرجعهای \cite{قندی1376,hayek1994} زیر نام نویسنده‌ی اصلی اثر مرتب میگردند.





\bibliography{Example-IMOS-2.095}


\end{document}