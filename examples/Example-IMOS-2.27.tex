\documentclass[a4paper,10pt]{article}

\def\examplenumber{مثال ۲.۲۷}


\usepackage[iran]{iran-bibtex}
\usepackage[colorlinks,citecolor=blue,breaklinks=true]{hyperref}

\usepackage[localise,perpagefootnote=on,computeautoilg=on]{xepersian}
\settextfont{Parsi Nevis}





\begin{document}
\title{شیوه‌ی استناددهی ایران}
\author{}
\date{}
\maketitle



\section*{\examplenumber}

مرجع \cite{کاظمی1380} یک کتاب فارسی با دو نویسنده است. مرجع \cite{walker1998} یک کتاب غیرفارسی با دو نویسنده است. مرجع \cite{marty1992} نیز یک کتاب غیرفارسی با دو نویسنده است با این تفاوت که نام فارسی نویسنده‌های آن وارد شده است.





\bibliography{Example-IMOS-2.27}


\end{document}