\documentclass[a4paper,10pt]{article}

\def\examplenumber{مثال ۲.۳۳}


\usepackage[iran]{iran-bibtex}
\usepackage[colorlinks,citecolor=blue,breaklinks=true]{hyperref}

\usepackage[localise,perpagefootnote=on,computeautoilg=on]{xepersian}
\settextfont{Parsi Nevis}





\begin{document}
\title{شیوه‌ی استناددهی ایران}
\author{}
\date{}
\maketitle



\section*{\examplenumber}

مرجع \cite{افراسیابی1378} کتاب فارسی‌ای است که پدیدآورنده‌ی آن ناشناس است و به کوشش شخص دیگری گردآوری شده است. در حالت کلی، اگر پدیدآورنده‌ی اثری نامشخص باشد، دو راه برای استناد به آن ممکن است: راه نخست آن است که نام پدیدآورنده وارد نشود \cite[مانند][]{هزارویک1378}؛ و راه دیگر آن است که به جای نام پدیدآورنده، «ناشناس» نوشته شود \cite[مانند][]{ناشناس1378,ناشناس1362}. با مرجعهای غیرفارسی نیز به همین‌گونه رفتار میشود.
\nocite{*}





\bibliography{Example-IMOS-2.33}


\end{document}