% !TEX TS-program = XeLaTeX
% Compilation procedure for this file:
% xelatex Example-IMOS-2.212-iran-unsrt.tex
% bibtex8 -W -c iran-bibtex-cp1256fa Example-IMOS-2.212-iran-unsrt
% xelatex Example-IMOS-2.212-iran-unsrt.tex
% xelatex Example-IMOS-2.212-iran-unsrt.tex

\documentclass[a4paper,10pt]{article}


\usepackage[iran-unsrt]{iran-bibtex}
\usepackage[colorlinks,citecolor=blue,breaklinks=true]{hyperref}

\usepackage[localise,perpagefootnote=on,computeautoilg=on]{xepersian}
\settextfont{Parsi Nevis}





\begin{document}
\title{شیوه‌ی استناددهی ایران
\LTRfootnote{ \tt \textbackslash usepackage[iran-unsrt]\{iran-bibtex\} } }
\author{}
\date{}
\maketitle



\section*{مثال ۲.۲۱۲: جزوه‌ی کلاسی}

زیربخش ۱. استاد نسخه‌ای از جزوه را در اختیار دانشجو قرار دهد که عنوان داشته باشد.\\
\cite{رضائیان1370الف}\\

زیربخش ۲. استاد نسخه‌ای از جزوه را در اختیار دانشجو قرار دهد که عنوان نداشته باشد.\\
\cite{رضائیان1370ب}\\

زیربخش ۳. استاد درس را بازگو کرده و دانشجو گفته‌ها، و تقریرها را یادداشت کند.\\
\cite{رضائیان1370پ}\\





\bibliography{Example-IMOS-2.212}


\end{document}