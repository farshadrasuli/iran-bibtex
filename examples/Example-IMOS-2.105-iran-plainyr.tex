% !TEX TS-program = XeLaTeX
% Compilation procedure for this file:
% xelatex Example-IMOS-2.105-iran-plainyr.tex
% bibtex8 -W -c iran-bibtex-cp1256fa Example-IMOS-2.105-iran-plainyr
% xelatex Example-IMOS-2.105-iran-plainyr.tex
% xelatex Example-IMOS-2.105-iran-plainyr.tex

\documentclass[a4paper,10pt]{article}


\usepackage[iran-plainyr]{iran-bibtex}
\usepackage[colorlinks,citecolor=blue,breaklinks=true]{hyperref}

\usepackage[localise,perpagefootnote=on,computeautoilg=on]{xepersian}
\settextfont{Parsi Nevis}





\begin{document}
\title{شیوه‌ی استناددهی ایران
\LTRfootnote{ \tt \textbackslash usepackage[iran-plainyr]\{iran-bibtex\} } }
\author{}
\date{}
\maketitle



\section*{مثال ۲.۱۰۵: یک جلد در چند کتاب (مجلد)}

\cite{جم1385}\\
\cite{harley1994}\\





\bibliography{Example-IMOS-2.105}


\end{document}